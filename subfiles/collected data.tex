\documentclass[../main.tex]{subfiles}
\begin{document}
Why RDP forms:
From \href{https://aip.scitation.org/doi/pdf/10.1063/1.5022817}{This article}:

In the gas-filled capillary discharge, the plasma channel is formed by the temperature profile during discharge, which is higher in the center due to the Ohmic heating effect of plasma and drops radially because the heating is balanced by heat conduction to the capillary wall. As the gas pressure is almost uniform, the radial plasma density has quasi-quadratic profile.
Refs: \href{https://journals.aps.org/pre/pdf/10.1103/PhysRevE.71.016401}{1} and \href{https://journals.aps.org/pre/pdf/10.1103/PhysRevE.65.016407}{2}.

In the formation phase, wall cooling shapes the channel.

The lower temperature at the wall has two opposing effects on $N_e$. The total particle density, and therefore the electron  density,  becomes higher because the pressure  remains roughly equal in the channel. However, the  lower temperature at the wall means that  plasma is not fully ionized near the wall.

The creation of pressure differences is amplified by redistribution of current. The  higher temperature in the center causes higher conductivity there, which causes a higher local current density. The net effect of this is an increased Ohmic dissipation density, causing further increases in central temperature and pressure.

\end{document}