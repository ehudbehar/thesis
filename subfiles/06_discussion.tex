\documentclass[../main.tex]{subfiles}

\begin{document}
In this thesis we studied the required conditions for guiding of high intensity lasers in the LWFA scheme in straight and curved gas--filled capillaries. The curved capillaries are intended to be a medium for composing several acceleration stages to propagate laser pulses over many Rayleigh lengths, in a "fish--bone" scheme.

We demonstrated a working set--up to produce jitter controlled plasma discharges in both straight and curved capillaries, achieving control on the time scale of a few nanoseconds,
\begin{equation*}
    	\tau_\text{jitter}\approx 5\pm 1\si{\ns}.
\end{equation*}
%The primary motivation is to guide a high intensity short laser pulse,
We characterized the plasma channel that forms in the capillary: The time window for which optical guiding exists was found to be $\sim$\SI{350}{\ns} after the plasma ignition, and to last
\begin{equation*}
    \Delta t_\text{channel}=50-150\ \si{\ns}.
\end{equation*}
%$$\Delta t_\text{channel}=\SIrange{50}{100}{\ns}.$$
Using a low intensity\todo{?} laser, a maximum radiation transmittance of 6.7 was observed.

A correlation between the applied voltage difference and the radiation transmittance was observed viz., a higher voltage difference implies a higher transmittance.

Formation of the plasma channel for the straight capillary was cross--checked by spectroscopy measurements. We observed radial parabolic density profile channel with an increase in the electron density of $$\Delta N_e =\SI{0.4e18}{\per\cubic\cm}$$ and radius
$$r_\text{ch}\approx \SI{50}{\um}$$ with on-axis electron density $$N_e(0)=\SI{0.4e18}{\per\cubic\cm}.$$

The mean plasma density across the longitudinal capillary dimension was found via spectroscopy measurement to be \SI{2.85e17}{\per\cubic\cm}.

When experimenting with a curved capillary, we couldn't repeat the same demonstration of optical guiding with the oscillator laser pulse--train. The system is jitter--controlled; The plasma observed to influence the laser radiation as the plasma becomes opaque in later stages of the capillary discharge, until the plasma diffuses to the vacuum chamber. However, in the time window at which we suppose optical guiding exist, we didn't observe the effect. 

In future experiments we will check for the guiding conditions of these gas filled capillaries and explore further the possibilities of incorporating these capillaries in a laser wakefield electron acceleration scheme. If successful, it would provide a possibility to accelerate electrons to \si{\tera\eV} energy levels at distances of tens of meters in comparison to many kilometers in classical accelerators.


%\begin{figure}
%    \centering
%    \includegraphics[width=\textwidth]{figures/Curved capillaries/coupling light to fiber.PNG}
%    \caption{\href{https://www.newport.com/t/fiber-optic-basics}{From here.}}
%    \label{fig:fiber}
%\end{figure}

\end{document}