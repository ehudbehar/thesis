\documentclass[../main.tex]{subfiles}

\begin{document}
In this thesis we studied various configurations of plasma channels generated in gas filled--capillaries for guiding of high intensity lasers in the LWFA scheme. The capillary configurations are intended to be a compact medium for composing several acceleration stages to propagate sequential laser pulses over many Rayleigh lengths, in a "fish--bone" scheme.

We demonstrated a working set up to produce jitter controlled plasma discharges in both straight and curved capillaries, achieving control on the time scale of a few nanoseconds,
\begin{equation*}
    	\tau_\text{jitter}\approx 3\pm 1\si{\ns}.
\end{equation*}
This achievement contributed to two scientific publications, namely  \cite{andBeharEhudLowCapillary} and \cite{andBeharEhudGasPulse}.\todo{New}
%$$\Delta t_\text{channel}=\SIrange{50}{100}{\ns}.$$

Using a low intensity, pulse--train laser beam passing through the capillary, we confirmed the formation of a plasma channel in the capillary during the plasma lifetime. We compared the signal voltage before the capillary discharge to the maximal signal voltage recorded during the discharge. The plasma channel forming in the capillary resulted optical guiding that caused an increase of the transmitted signal in this specific time window, at which we observed an increase by as much as $\sim \times 8$.

The plasma channel appeared to form in a consistent timing, about \SI{250}{\ns} after the capillary ignition, and to last 
\begin{equation*}
    \Delta t_\text{channel}=50-150\ \si{\ns}.
\end{equation*}

We observed a radial, parabolic hollow density profile in the plasma channel, with an electron density depth of about 
$$\Delta N_e \approx\SI{0.4e18}{\per\cubic\cm}$$
and channel radius
$$r_\text{ch}\approx \SI{50}{\um}.$$

The on-axis electron density was measured to be on the order of $$N_e(0)\approx \SI{0.4e18}{\per\cubic\cm}.$$

%We also imaged, with the same considerations of the emitted H\textsubscript{$\alpha$} line as an indicator, the \SI{5}{\cm} longitudinal capillary dimension to estimate the mean plasma density during a discharge. The plasma density was found to be homogeneous across the length, with a mean, $\bar{N}_e$, on the order of
%\begin{equation*}
%    \bar{N}_e \approx \SI{0.28e18}{\per\cubic\cm},
%\end{equation*}
% and a standard deviation of about \SI{0.33e17}{\per\cubic\cm}.

%When experimenting with a curved capillary, we couldn't repeat the same demonstration of optical guiding with the oscillator laser pulse--train.
Experimentation with the curved capillary showed similar results in regard to jitter when igniting the plasma discharge by use of the Nd:Yag igniting pulse, a jitter on the order of a few nanoseconds.

We arranged a similar set--up using the oscillator pulse--train, traveling through the capillary, to observe the plasma channel that forms in the capillary. The plasma observed to influence the laser radiation by becoming opaque in the later stages of the capillary discharge, until the plasma diffuses to the vacuum chamber.

In future experiments we will check for the guiding conditions of the curved gas filled capillaries and explore further the possibilities of incorporating these capillaries in a laser wakefield electron acceleration scheme. If successful, it would provide a possibility to accelerate electrons to \si{\tera\eV} energy levels at distances of tens of meters in comparison to tens of kilometers in classical accelerators.

\printbibliography

\end{document}