\documentclass[../main.tex]{subfiles}
\begin{document}
Results of experimental work with 3D printed gas filled plasma-discharge capillaries are presented, as a part of the laser wakefield acceleration (LWFA) scheme.

We demonstrate a synchronization capability in the scale of 1 ns using straight and curved capillaries. 

This high precession opens a hatch into the possibility of sufficient synchronization between three systems -- synchronization between the plasma generation, the incident ultra-short high intensity laser and the injection of the electrons.

We also demonstrate the formation of a plasma channel in the (straight?) capillary, resulting optical guiding, with an 84 MHz Ti:Sa oscillator laser, for the purpose of extending the propagation distances of intense laser pulses over many Rayleigh lengths, in a "fish--bone" scheme.

The verification of a preformed plasma channel with a hollow, parabolic plasma density profile using a temporally resolved plasma spectroscopy measurement is also presented.


תוצאות של עבודה ניסיונית עם התפרקות בקפילרה מודפסת בתלת ממד
\end{document}